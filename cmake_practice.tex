\documentclass[oneside, 12pt]{book}
\usepackage[utf8]{inputenc}
\usepackage[colorlinks,linkcolor=black,anchorcolor=black,citecolor=black]{hyperref}
\usepackage{xeCJK}
\usepackage{listings}
\usepackage{bm}
\usepackage{xcolor}
\usepackage{minted}
\usepackage[left=1.5cm, right=1.5cm]{geometry}
\usepackage{fontspec} 
\usepackage[normalem]{ulem}
\lstdefinestyle{lfonts}{
  basicstyle= \footnotesize\ttfamily,
  stringstyle= \color{purple},
  keywordstyle = \color{blue!60!black}\bfseries,
  commentstyle = \color{olive}\scshape,
}
\lstdefinestyle{lnumbers}{
  numbers=left,
  numberstyle=\tiny,
  numbersep= 1em,
  firstnumber= 1,
  stepnumber = 1,
}
\lstdefinestyle{llayout}{
  breaklines       = true,
  tabsize          = 2,
  columns          = flexible,
}
\lstdefinestyle{lgeometry}{
  xleftmargin      = 20pt,
  xrightmargin     = 0pt,
  frame            = tb,
  framesep         = \fboxsep,
  framexleftmargin = 20pt,
}
\lstdefinestyle{lgeneral}{
  style = lfonts,
  style = lnumbers,
  style = llayout,
  style = lgeometry,
}
\lstdefinestyle{c++}{
    language = {c++},
    style    = lgeneral,
}
\setCJKmainfont{PingFangSC-Regular}
\setCJKmonofont{PingFangSC-Regular}
\setmainfont{Times New Roman}
\font\subtitleFont=NovaMono at 40pt
\font\authorFont=NovaMono at 16pt
\font\NovaMonoFont=NovaMono
\title{\fontsize{50}{60}\selectfont Cmake 实践\\ \subtitleFont{Cmake Practice}}
\author{\authorFont{Cjacker \& VIXNESS.meolin}\\ \\ \NovaMonoFont{vix2018@gmail.com}}
\date{\NovaMonoFont 2019}
\begin{document}
\begin{titlepage}
  \maketitle
\end{titlepage}
\tableofcontents
\setcounter{chapter}{0}
	\renewcommand{\thechapter}{\Roman{chapter}}
  \chapter{序 I}
  Cmake 已经开发了 5,6 年的时间, 如果没有 KDE4, 也许不会有人或者 Linux 发行版 本重视 cmake, 因为除了 Kitware 似乎没有人使用它.通过 KDE4 的选型和开发, cmake 逐渐进入了人们的视线, 在实际的使用过程中, cmake 的优势也逐渐的被大家所认识, 至 少 KDE 的开发者们给予了 cmake 极高的评价, 同时庞大的 KDE 项目使用 cmake 来作为构 建工具也证明了 cmake 的可用性和大项目管理能力.\\ \\
所以, cmake 应该感谢 KDE, 也正因为如此, cmake 的开发者投入了 KDE 从 autotools 到 cmake 的迁移过程中, 并相当快速和顺利的完成了迁移, 现在整个 KDE4 开 发版本全部使用 cmake 构建.\\ \\
这也是促使我们学习 cmake 的原因, 首先 cmake 被接受并成功应用, 其次, cmake 的优势在实际使用中不断的体现出来.\\ \\
我们为什么不来认识一下这款优秀的工程构建工具呢?\\
在 2006 年 KDE 大会, 听 cmake 开发者当面介绍了 cmake 之后, 我就开始关注 cmake, 并将 cmake 纳入了 Everest 发行版, 作为系统默认组件.最近 QT-4.3 也正式进 入了 Everest 系统, 为 KDE4 构建完成了准备工作.\\\\
但是, 在学习 cmake 的过程中, 发现官方的文档非常的少, 而且错误也较多, 比如:\\\\
在介绍模块Find<Name>编写的文档中, 模块名称为FOO, 但是后面却出现了Foo\_FIND\_QUIETLY的定义,  这显然是错误的, 这样的定义永远不可能有效, 正确的定义是FOO\_FIND\_QUIETLY. 
种种原因, 促使我开始写一份"面向使用和实用"的cmake文档, 也就是本教程《Cmake 实践》(Cmake Practice)
本文档是边学习边编写的成果, 更像是一个学习笔记和 Tutorial, 因此难免有失误 或者理解不够透彻的地方, 比如, 我仍然不能理解为什么绝大部分使用变量的情况要通过\$\{\}
引用, 而在 IF 语句中却必须直接使用变量名.也希望能够有 cmake 的高手来指点迷津.\\\\
The 'IF(var)' or 'IF(NOT var)' command expects 'var' to be the name of a variable. This is stated in CMake's manual. So, for your situation 'IF(\$\{libX\})'
 is the same as 'IF(/usr/lib/xorg)' and then CMake will check the value of the variable named '/usr/lib/xorg'. 也就是说 IF 需要的是变量名而不是变量值.\\\\
这个文档是开放的, 开放的目的是为了让更多的人能够读到并且能够修改, 任何人都 可以对它作出修改和补充, 但是, 为了大家都能够获得你关于 cmake 的经验和积累, 如果 你现错误或者添加了新内容后, 请务必 CC 给我一份, 让我们共同把 cmake 掌握的更好.\\\\
 \chapter{序 II}
  最近开始接触cmake, 在学习时看到了这本书, 收益匪浅. 但书中存在着一些错误
  因为无法联系到原作者(若有人知道原作者联系方式请告知我, 感激不尽), 所以我
  对这本书进行订正与重新排版. 再一次感谢原作者.
\setcounter{chapter}{0}
  \renewcommand{\thechapter}{\arabic{chapter}}
  \chapter{初识 Cmake}
  \framebox[\linewidth]{Cmake 不再使你在构建项目时郁闷地想自杀了.——一位 KDE 开发者} 
  \section{背景知识}
    cmake是kitware 公司以及一些开源开发者在开发几个工具套件(VTK)的过程中衍 生品,最终形成体系,成为一个独立的开放源代码项目.项目的诞生时间是 2001 年.其官 方网站是 www.cmake.org,可以通过访问官方网站获得更多关于 cmake 的信息.cmake 的流行其实要归功于 KDE4 的开发(似乎跟当年的 svn 一样,KDE 将代码仓库从 CVS 迁移到 SVN,同时证明了 SVN 管理大型项目的可用性),在 KDE 开发者使用了近 10 年 autotools 之后,他们终于决定为 KDE4 选择一个新的工程构建工具,其根本原因用 KDE 开发者的话来 说就是:只有少数几个"编译专家"能够掌握 KDE 现在的构建体系 (admin/Makefile.common),在经历了 unsermake, scons 以及 cmake 的选型和尝 试之后,KDE4 决定使用 cmake 作为自己的构建系统.在迁移过程中,进展异常的顺利,并 获得了 cmake 开发者的支持.所以,目前的 KDE4 开发版本已经完全使用 cmake 来进行构 建.像 kdesvn,rosegarden 等项目也开始使用 cmake,这也注定了 cmake 必然会成为 一个主流的构建体系.

\section{特点}
Cmake 的特点主要有: \\
\begin{enumerate}
  \item 开放源代码,使用类 BSD 许可发布.http://cmake.org/HTML/Copyright.html
  \item 跨平台,并可生成 native 编译配置文件,在 Linux/Unix 平台,生成 makefile,在苹果平台可以生成xcode,在 Windows 平台,可以生成 MSVC 的工程文件.
  \item 能够管理大型项目,KDE4 就是最好的证明.
  \item 简化编译构建过程和编译过程.Cmake 的工具链非常简单:cmake+make.
  \item 高效虑,按照 KDE 官方说法,CMake 构建 KDE4 的 kdelibs 要比使用 autotools 来 构建 KDE3.5.6 的 kdelibs 快 40\%,主要是因为 Cmake 在工具链中没有 libtool.
  \item 可扩展,可以为 cmake 编写特定功能的模块,扩充 cmake 功能.
\end{enumerate}

\section{问题,难道就没有问题?}
\begin{enumerate}
  \item cmake 很简单,但绝对没有听起来或者想象中那么简单.

  \item cmake 编写的过程实际上是编程的过程,跟以前使用 autotools 一样,不过你需要编 写的是 CMakeLists.txt(每个目录一个),使用的是"cmake 语言和语法".

  \item cmake 跟已有体系的配合并不是特别理想,比如 pkgconfig,您在实际使用中会有所 体会,虽然有一些扩展可以使用,但并不理想.
\end{enumerate}
\section{个人的建议}
\begin{enumerate}
  \item 如果你没有实际的项目需求,那么看到这里就可以停下来了,因为 cmake 的学习过程就 是实践过程,没有实践,读的再多几天后也会忘记.

  \item 如果你的工程只有几个文件,直接编写 Makefile 是最好的选择.

  \item 如果使用的是 C/C++/Java 之外的语言,请不要使用 cmake(至少目前是这样)

  \item 如果你使用的语言有非常完备的构建体系,比如 java 的 ant,也不需要学习 cmake, 虽然有成功的例子,比如 QT4.3 的 csharp 绑定 qyoto.
  \item 如果项目已经采用了非常完备的工程管理工具,并且不存在维护问题,没有必要迁移到 cmake 
  \item 如果仅仅使用 qt 编程,没有必要使用 cmake,因为 qmake 管理 Qt 工程的专业性和自 动化程度比 cmake 要高很多.
\end{enumerate}
\chapter{安装 Cmake}
\section{还需要安装吗?}
cmake 目前已经成为各大 Linux 发行版提供的组件,比如 Everest 直接在系统中包含, Fedora 在 extra 仓库中提供,所以,需要自己动手安装的可能性很小.\\
如果你使用的操 作系统(比如 Windows 或者某些 Linux 版本)没有提供 cmake 或者包含的版本较旧,建议 你直接从 cmake 官方网站下载安装.
\begin{center}
  \framebox{{\centering http://www.cmake.org/HTML/Download.html}}\par
\end{center}
在这个页面,提供了源代码的下载以及针对各种不同操作系统的二进制下载,可以选择适合 自己操作系统的版本下载安装.因为各个系统的安装方式和包管理格式有所不同,在此就不 再赘述了,相信一定能够顺利安装 cmake.

\chapter{初试Cmake —— Cmake 的Hello World}
\framebox[\linewidth]{Hello World, 你好~世界}\\
本节选择了一个最简单的例子 Helloworld 来演练一下 cmake 的完整构建过程,本节并不 会深入的探讨 cmake,仅仅展示一个简单的例子,并加以粗略的解释.
\section{准备工作}
首先,在/backup目录建立一个cmake 目录,用来放置我们学习过程中的所有练习.
\begin{minted}[mathescape,numbersep=5pt,gobble=2,frame=lines,framesep=2mm]{shell}
  mkdir -p /backup/cmake
\end{minted}
以后我们所有的 cmake 练习都会放在/backup/cmake 的子目录下(你也可以自行安排目录, 这个并不是限制,仅仅是为了叙述的方便)
然后在 cmake 建立第一个练习目录 t1
\begin{minted}[mathescape,linenos,numbersep=5pt,gobble=2,frame=lines,framesep=2mm]{shell}
  cd /backup/cmake 
  mkdir t1 
  cd t1
\end{minted}
\newpage
\noindent 在t1 目录建立main.c 和CMakeLists.txt (注意文件名大小写):\\
main.c 文件内容:
\begin{minted}[mathescape,linenos,numbersep=5pt,gobble=2,frame=lines,framesep=2mm]{c}
  #include <stdio.h> 
  int main() {
    printf("Hello World from t1 Main!\n");
    return 0; 
  }
\end{minted}
CmakeLists.txt 文件内容:
\begin{minted}[mathescape,linenos,numbersep=5pt,gobble=2,frame=lines,framesep=2mm]{cmake}
  PROJECT (HELLO) 
  SET(SRC_LIST main.cpp) 
  MESSAGE(STATUS "This is BINARY dir " ${HELLO_BINARY_DIR}) 
  MESSAGE(STATUS "This is SOURCE dir "${HELLO_SOURCE_DIR}) 
  ADD_EXECUTABLE(hello ${SRC_LIST})
\end{minted}

\section{开始构建}
所有的文件创建完成后,t1 目录中应该存在 main.c 和 CMakeLists.txt 两个文件 接下来我们来构建这个工程,在这个目录运行:
\begin{minted}[mathescape,numbersep=5pt,gobble=2,frame=lines,framesep=2mm]{shell}
  cmake . #注意命令后面的点号,代表本目录
\end{minted}

输出大概是这个样子:
\begin{minted}[mathescape,numbersep=5pt,gobble=2,frame=lines,framesep=2mm]{shell}
  -- Check for working C compiler: /usr/bin/gcc
  -- Check for working C compiler: /usr/bin/gcc -- works
  -- Check size of void*
  -- Check size of void* - done
  -- Check for working CXX compiler: /usr/bin/c++
  -- Check for working CXX compiler: /usr/bin/c++ -- works
  -- This is BINARY dir /backup/cmake/t1
  -- This is SOURCE dir /backup/cmake/t1
  -- Configuring done
  -- Generating done
  -- Build files have been written to: /backup/cmake/t1
\end{minted}
再让我们看一下目录中的内容: 你会发现,系统自动生成了: CMakeFiles, CMakeCache.txt, cmake\_install.cmake 等文件,并且生成了 Makefile.\\
现在不需要理会这些文件的作用,以后你也可以不去理会.最关键的是,它自动生成了 Makefile.\\
然后进行工程的实际构建,在这个目录输入 make 命令,大概会得到如下的彩色输出:
\begin{minted}[mathescape,numbersep=5pt,gobble=2,frame=lines,framesep=2mm]{shell}
  Scanning dependencies of target hello 
  [100%] Building C object CMakeFiles/hello.dir/main.o Linking C executable hello 
  [100%] Built target hello
\end{minted}
如果你需要看到 make 构建的详细过程,可以使用 make VERBOSE=1 或者 VERBOSE=1 make 命令来进行构建.\\
这时候,我们需要的目标文件 hello 已经构建完成,位于当前目录,尝试运行一下: ./hello 得到输出: Hello World from Main 恭喜您,到这里为止您已经完全掌握了 cmake 的使用方法.
\section{简单的解释}
我们来重新看一下 CMakeLists.txt,这个文件是 cmake 的构建定义文件,文件名 是大小写相关的,如果工程存在多个目录,需要确保每个要管理的目录都存在一个 CMakeLists.txt.(关于多目录构建,后面我们会提到,这里不作过多解释).\\
上面例子中的 CMakeLists.txt 文件内容如下:
\begin{minted}[mathescape,linenos,numbersep=5pt,gobble=2,frame=lines,framesep=2mm]{cmake}
  PROJECT (HELLO) 
  SET(SRC_LIST main.cpp) 
  MESSAGE(STATUS "This is BINARY dir " ${HELLO_BINARY_DIR}) 
  MESSAGE(STATUS "This is SOURCE dir "${HELLO_SOURCE_DIR}) 
  ADD_EXECUTABLE(hello ${SRC_LIST})
\end{minted}
\subsection{PROJECT 指令的语法}
\begin{minted}[mathescape,numbersep=5pt,gobble=2,frame=lines,framesep=2mm]{cmake}
PROJECT(projectname [CXX] [C] [Java]) 
\end{minted}
你可以用这个指令定义工程名称,并可指定工程支持的语言,支持的语言列表是可以忽略的, 默认情况表示支持所有语言.这个指令隐式的定义了两个 cmake 变量: 
\mintinline{cmake}/<projectname>_BINARY_DIR/ 以及\mintinline{cmake}/projectname>_SOURCE_DIR/,这里就是 \mintinline{cmake}/HELLO_BINARY_DIR/和 
\mintinline{cmake}/HELLO_SOURCE_DIR/(所以 CMakeLists.txt 中两个 \mintinline{cmake}/MESSAGE/指令可以直接使用了这两个变量),因为采用的是内部编译,两个变量目前指的都是工程所
在路径/backup/cmake/t1,后面我们会讲到外部编译,两者所指代的内容会有所不同.\\
同时 cmake 系统也帮助我们预定义了 \mintinline{cmake}/PROJECT_BINARY_DIR/和 \mintinline{cmake}/PROJECT_SOURCE_DIR/变量,他们的值分别跟 \mintinline{cmake}/HELLO_BINARY_DIR/与 \mintinline{cmake}/HELLO_SOURCE_DIR/ 一致.\\
为了统一起见,建议以后直接使用 \mintinline{cmake}/PROJECT_BINARY_DIR/,\mintinline{cmake}/PROJECT_SOURCE_DIR/,即 使修改了工程名称,也不会影响这两个变量.如果使用了 \mintinline{cmake}/<projectname>_SOURCE_DIR/,修改工程名称后,需要同时修改这些变量.
\subsection{SET 指令的语法}
\begin{minted}[mathescape,numbersep=5pt,gobble=2,frame=lines,framesep=2mm]{cmake}
SET(VAR [VALUE] [CACHE TYPE DOCSTRING [FORCE]]) 
\end{minted}
现阶段,你只需要了解 SET 指令可以用来显式的定义变量即可. 比如我们用到的是 \mintinline{cmake}/SET(SRC_LIST main.c)/,
如果有多个源文件,也可以定义成: \mintinline{cmake}/SET(SRC_LIST main.c t1.c t2.c)/.

\subsection{MESSAGE 指令的语法}
\begin{minted}[mathescape,numbersep=5pt,gobble=2,frame=lines,framesep=2mm]{cmake}
  MESSAGE([SEND_ERROR | STATUS | FATAL_ERROR] "message to display" ...)
\end{minted}
这个指令用于向终端输出用户定义的信息,包含了三种类型: SEND\_ERROR,产生错误,生成过程被跳过. \mintinline{cmake}/SATUS/,输出前缀为—的信息. \mintinline{cmake}/FATAL_ERROR/,立即终止所有 cmake 过程.\\
我们在这里使用的是 \mintinline{cmake}/STATUS/ 信息输出,演示了由 PROJECT 指令定义的两个隐式变量 \mintinline{cmake}/HELLO_BINARY_DIR/ 和 \mintinline{cmake}/HELLO_SOURCE_DIR/.\\
\mintinline{cmake}/ADD_EXECUTABLE(hello ${SRC_LIST})/ 定义了这个工程会生成一个文件名为 hello 的可执行文件,相关的源文件是 \mintinline{cmake}/SRC_LIST/中 定义的源文件列表, 本例中你也可以直接写成 \mintinline{cmake}/ADD_EXECUTABLE(hello main.c)/.\\
在本例我们使用了\mintinline{cmake}/${}/来引用变量,这是 cmake 的变量引用方式,但是,有一些例外,比 如在 IF 控制语句,变量是直接使用变量名引用,而不需要\mintinline{cmake}/${}/.
如果使用了\mintinline{cmake}/${}/去应用变量,其实IF 会去判断名为
\mintinline{cmake}/${}/所代表的值的变量,那当然是不存在的了.\\
将本例改写成一个最简化的 CMakeLists.txt: 
\begin{minted}[mathescape,numbersep=5pt,gobble=2,frame=lines,framesep=2mm]{cmake}
  PROJECT(HELLO) ADD_EXECUTABLE(hello main.c)
\end{minted}
\section{基本语法规则}
前面提到过,cmake 其实仍然要使用"cmake 语言和语法"去构建,上面的内容就是所谓的 "cmake 语言和语法",最简单的语法规则是: 
\begin{enumerate}
    \item 变量使用\mintinline{cmake}/${}/方式取值,但是在 IF 控制语句中是直接使用变量名.
    \item 指令(参数 1 参数 2...) 参数使用括弧括起,参数之间使用空格或分号分开. 以上面的\mintinline{cmake}/ADD_EXECUTABLE/指令为例,如果存在另外一个func.c源文件,就要写成: \mintinline{cmake}/ADD_EXECUTABLE(hello main.c func.c)/或者 \mintinline{cmake}/ADD_EXECUTABLE(hello main.c;func.c)/
    \item 指令是大小写无关的,参数和变量是大小写相关的.但,推荐你全部使用大写指令.
\end{enumerate}
上面的\mintinline{cmake}/MESSAGE/指令我们已经用到了这条规则:\\ 
\mintinline{cmake}/MESSAGE(STATUS "This is BINARY dir" ${HELLO_BINARY_DIR})/ 
也可以写成:\\ 
\mintinline{cmake}/MESSAGE(STATUS "This is BINARY dir ${HELLO_BINARY_DIR}")/
这里需要特别解释的是作为工程名的HELLO 和生成的可执行文件hello 是没有任何关系的. hello 定义了可执行文件的文件名,你完全可以写成: \mintinline{cmake}/ADD_EXECUTABLE(t1 main.c)/ 编译后会生成一个 t1 可执行文件.
\section{关于语法的疑惑}
cmake 的语法还是比较灵活而且考虑到各种情况,比如\mintinline{cmake}/SET(SRC_LIST main.c)/写成
\mintinline{cmake}/SET(SRC_LIST "main.c")/
是没有区别的,但是假设一个源文件的文件名是 fu nc.c(文件名中间包含了空格). 这时候就必须使用双引号,如果写成了\mintinline{cmake}/SET(SRC_LIST fu nc.c)/,就会出现错误,提示 你找不到 fu 文件和 nc.c 文件.这种情况,就必须写成:\\
\mintinline{cmake}/SET(SRC_LIST "fu nc.c")/
此外,你可以可以忽略掉 source 列表中的源文件后缀,比如可以写成 \mintinline{cmake}/ADD_EXECUTABLE(t1 main)/,cmake 会自动的在本目录查找 main.c 或者 main.cpp 等,当然,最好不要偷这个懒,以免这个目录确实存在一个 main.c 一个 main.\\
同时参数也可以使用分号来进行分割. 下面的例子也是合法的: \mintinline{cmake}/ADD_EXECUTABLE(t1 main.c t1.c)/可以写成\mintinline{cmake}/ADD_EXECUTABLE(t1 main.c;t1.c)/.\\
我们只需要在编写 CMakeLists.txt 时注意形成统一的风格即可.
\section{清理工程}
跟经典的 autotools 系列工具一样,运行: 
\begin{minted}[mathescape,numbersep=5pt,gobble=2,frame=lines,framesep=2mm]{shell}
  make clean 
\end{minted}
即可对构建结果进行清理.
\section{问题?问题!}
“我尝试运行了\mintinline{shell}/make distclean/,这个指令一般用来清理构建过程中产生的中间文件的, 如果要发布代码,必然要清理掉所有的中间文件,但是为什么在 cmake 工程中这个命令是 无效的?” 是的,cmake 并不支持 make distclean,关于这一点,官方是有明确解释的:\\
因为 CMakeLists.txt 可以执行脚本并通过脚本生成一些临时文件,但是却没有办法来跟 踪这些临时文件到底是哪些.因此,没有办法提供一个可靠的\mintinline{shell}/make distclean/方案.\\ \\
Some build trees created with GNU autotools have a "make distclean" target that cleans the build and also removes Makefiles and other parts of the generated build system. CMake does not generate a "make distclean" target because CMakeLists.txt files can run scripts and arbitrary commands; CMake has no way of tracking exactly which files are generated as part of running CMake. Providing a distclean target would give users the false impression that it would work as expected. (CMake does generate a "make clean" target to remove files generated by the compiler and linker.)
A "make distclean" target is only necessary if the user performs an in-source build. CMake supports in-source builds, but we strongly encourage users to adopt the notion of an out-of-source build. Using a build tree that is separate from the source tree will prevent CMake from generating any files in the source tree. Because CMake does not change the source tree, there is no need for a distclean target. One can start a fresh build by deleting the build tree or creating a separate build tree.\\ \\

同时,还有另外一个非常重要的提示,就是:我们刚才进行的是内部构建(in-source build),而 cmake 强烈推荐的是外部构建(out-of-source build).
\section{内部构建与外部构建}
上面的例子展示的是“内部构建”,相信看到生成的临时文件比您的代码文件还要多的时候, 估计这辈子你都不希望再使用内部构建.\\
举个简单的例子来说明外部构建,以编译 wxGTK 动态库和静态库为例,在 Everest 中打包 方式是这样的:
\begin{enumerate}
  \item 解开 wxGTK.
  \item 在其中建立 static 和 shared 目录.
  \item 进入 static 目录,运行../configure –enable-static;make 会在 static 目录生 成 wxGTK 的静态库.
  \item 进入 shared 目录,运行../configure –enable-shared;make 就会在 shared 目录 生成动态库.
\end{enumerate}
这就是外部编译的一个简单例子.\\

对于 cmake,内部编译上面已经演示过了,它生成了一些无法自动删除的中间文件,所以, 引出了我们对外部编译的探讨,外部编译的过程如下:
\begin{enumerate}
  \item 首先,请清除 t1 目录中除 main.c CmakeLists.txt 之外的所有中间文件,最关键 的是 CMakeCache.txt.
  \item 在 t1 目录中建立 build 目录,当然你也可以在任何地方建立 build 目录,不一定必 须在工程目录中.
  \item 进入 build 目录,运行 cmake ..(注意,..代表父目录,因为父目录存在我们需要的 CMakeLists.txt,如果你在其他地方建立了 build 目录,需要运行 cmake <工程的全 路径>),查看一下 build 目录,就会发现了生成了编译需要的 Makefile 以及其他的中间 文件.
  \item 运行 make 构建工程,就会在当前目录(build 目录)中获得目标文件 hello.
\end{enumerate}
上述过程就是所谓的 out-of-source 外部编译,一个最大的好处是,对于原有的工程没 有任何影响,所有动作全部发生在编译目录.通过这一点,也足以说服我们全部采用外部编 译方式构建工程. 这里需要特别注意的是:

通过外部编译进行工程构建,\mintinline{cmake}/HELLO_SOURCE_DIR/ 仍然指代工程路径,即 /backup/cmake/t1 而 \mintinline{cmake}/HELLO_BINARY_DIR/ 则指代编译路径,即/backup/cmake/t1/build
\section{小结}
本小节描述了使用 cmake 构建 Hello World 程序的全部过程,并介绍了三个简单的指令:
\mintinline{cmake}/PROJECT/, \mintinline{cmake}/MESSAGE/, \mintinline{cmake}/ADD_EXECUTABLE/以及变量调用的方法,同时提及了两个隐式变量 \mintinline{cmake}/<projectname>_SOURCE_DIR/\\
及\mintinline{cmake}/<projectname>_BINARY_DIR/,演示了变量调用的方法,从这个过程来看,有些开发者可能会想,这实在比我直接写 Makefile 要复杂多了, 甚至我都可以不编写 Makefile,直接使用 gcc main.c 即可生成需要的目标文件。是的, 正如第一节提到的,如果工程只有几个文件,还是直接编写 Makefile 最简单。但是,kdelibs 压缩包达到了50多M,您认为使用什么方案会更容易一点呢?\\

下一节,我们的任务是让 HelloWorld 看起来更像一个工程。
\end{document}
